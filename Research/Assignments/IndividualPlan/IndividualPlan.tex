\documentclass{sigchi}

% Use this section to set the ACM copyright statement (e.g. for
% preprints).  Consult the conference website for the camera-ready
% copyright statement.

% Copyright
\CopyrightYear{2020}
%\setcopyright{acmcopyright}
\setcopyright{acmlicensed}
%\setcopyright{rightsretained}
%\setcopyright{usgov}
%\setcopyright{usgovmixed}
%\setcopyright{cagov}
%\setcopyright{cagovmixed}
% DOI
\doi{https://doi.org/10.1145/3313831.XXXXXXX}
% ISBN
\isbn{978-1-4503-6708-0/20/04}
%Conference
\conferenceinfo{CHI'20,}{April  25--30, 2020, Honolulu, HI, USA}
%Price
\acmPrice{\$15.00}

% Use this command to override the default ACM copyright statement
% (e.g. for preprints).  Consult the conference website for the
% camera-ready copyright statement.

%% HOW TO OVERRIDE THE DEFAULT COPYRIGHT STRIP --
%% Please note you need to make sure the copy for your specific
%% license is used here!
% \toappear{
% Permission to make digital or hard copies of all or part of this work
% for personal or classroom use is granted without fee provided that
% copies are not made or distributed for profit or commercial advantage
% and that copies bear this notice and the full citation on the first
% page. Copyrights for components of this work owned by others than ACM
% must be honored. Abstracting with credit is permitted. To copy
% otherwise, or republish, to post on servers or to redistribute to
% lists, requires prior specific permission and/or a fee. Request
% permissions from \href{mailto:Permissions@acm.org}{Permissions@acm.org}. \\
% \emph{CHI '16},  May 07--12, 2016, San Jose, CA, USA \\
% ACM xxx-x-xxxx-xxxx-x/xx/xx\ldots \$15.00 \\
% DOI: \url{http://dx.doi.org/xx.xxxx/xxxxxxx.xxxxxxx}
% }

% Arabic page numbers for submission.  Remove this line to eliminate
% page numbers for the camera ready copy
% \pagenumbering{arabic}

% Load basic packages
\usepackage{balance}       % to better equalize the last page
\usepackage{graphics}      % for EPS, load graphicx instead 
\usepackage[T1]{fontenc}   % for umlauts and other diaeresis
\usepackage{txfonts}
\usepackage{mathptmx}
\usepackage[pdflang={en-US},pdftex]{hyperref}
\usepackage{color}
\usepackage{booktabs}
\usepackage{textcomp}

% Some optional stuff you might like/need.
\usepackage{microtype}        % Improved Tracking and Kerning
% \usepackage[all]{hypcap}    % Fixes bug in hyperref caption linking
\usepackage{ccicons}          % Cite your images correctly!
% \usepackage[utf8]{inputenc} % for a UTF8 editor only

% If you want to use todo notes, marginpars etc. during creation of
% your draft document, you have to enable the "chi_draft" option for
% the document class. To do this, change the very first line to:
% "\documentclass[chi_draft]{sigchi}". You can then place todo notes
% by using the "\todo{...}"  command. Make sure to disable the draft
% option again before submitting your final document.
\usepackage{todonotes}

% Paper metadata (use plain text, for PDF inclusion and later
% re-using, if desired).  Use \emtpyauthor when submitting for review
% so you remain anonymous.
\def\plaintitle{Individual Plan - Degree Project}
\def\subplaintitle{
  Reducing Sedentary Behavior for Software Engineers: Identified Performance Issues and Benefits Using a Visual Programming Language Inside Virtual Reality
}
\def\plainauthor{First Author, Second Author, Third Author,
  Fourth Author, Fifth Author, Sixth Author}
\def\emptyauthor{}
\def\plainkeywords{Authors' choice; of terms; separated; by
  semicolons; include commas, within terms only; this section is required.}
\def\plaingeneralterms{Documentation, Standardization}

% llt: Define a global style for URLs, rather that the default one
\makeatletter
\def\url@leostyle{%
  \@ifundefined{selectfont}{
    \def\UrlFont{\sf}
  }{
    \def\UrlFont{\small\bf\ttfamily}
  }}
\makeatother
\urlstyle{leo}

% To make various LaTeX processors do the right thing with page size.
\def\pprw{8.5in}
\def\pprh{11in}
\special{papersize=\pprw,\pprh}
\setlength{\paperwidth}{\pprw}
\setlength{\paperheight}{\pprh}
\setlength{\pdfpagewidth}{\pprw}
\setlength{\pdfpageheight}{\pprh}

% Make sure hyperref comes last of your loaded packages, to give it a
% fighting chance of not being over-written, since its job is to
% redefine many LaTeX commands.
\definecolor{linkColor}{RGB}{6,125,233}
\hypersetup{%
  pdftitle={\plaintitle},
% Use \plainauthor for final version.
%  pdfauthor={\plainauthor},
  pdfauthor={\emptyauthor},
  pdfkeywords={\plainkeywords},
  pdfdisplaydoctitle=true, % For Accessibility
  bookmarksnumbered,
  pdfstartview={FitH},
  colorlinks,
  citecolor=black,
  filecolor=black,
  linkcolor=black,
  urlcolor=linkColor,
  breaklinks=true,
  hypertexnames=false
}

% create a shortcut to typeset table headings
% \newcommand\tabhead[1]{\small\textbf{#1}}

% End of preamble. Here it comes the document.
\begin{document}

\title{%
  \plaintitle \\
  \large \subplaintitle
}

\numberofauthors{3}
\author{
  \alignauthor{Adam Jonsson\\
  \affaddr{Stockholm, Sweden}\\
  \email{adajon@kth.se}}\\
}

\maketitle


% ACM Classfication

\begin{CCSXML}
<ccs2012>
<concept>
<concept_id>10003120.10003121</concept_id>
<concept_desc>Human-centered computing~Human computer interaction (HCI)</concept_desc>
<concept_significance>500</concept_significance>
</concept>
<concept>
<concept_id>10003120.10003121.10003125.10011752</concept_id>
<concept_desc>Human-centered computing~Haptic devices</concept_desc>
<concept_significance>300</concept_significance>
</concept>
<concept>
<concept_id>10003120.10003121.10003122.10003334</concept_id>
<concept_desc>Human-centered computing~User studies</concept_desc>
<concept_significance>100</concept_significance>
</concept>
</ccs2012>
\end{CCSXML}

\ccsdesc[500]{Human-centered computing~Human computer interaction (HCI)}
\ccsdesc[300]{Human-centered computing~Haptic devices}
\ccsdesc[100]{Human-centered computing~User studies}

\section{Project Information}

\subsubsection{Preliminary title} 
\subplaintitle

\subsubsection{Student} 
Adam Jonsson - adajon@kth.se

\subsubsection{Examiner} 
Cristian Bogdan - cristi@kth.se

\subsubsection{Supervisor} 
Martin Hedlund - marthed@kth.se

\subsubsection{Current Date} 
2022-01-11

\subsubsection{Keywords} 
Visual Programming Language, Virtual Reality, Performance, Health.

\section{Background And Objective}
% Sedentary is bad
% There exist research about it
% VR is good health wise
% VR is bad performance wise
% Visual programming language may be useful for VR
% Identify performance differences between desktop and VR, what and why.
% 

Profession that is primarily conducted sedentary have increased due to a growth of office related occupation \cite{parry_contribution_2013}. One such profession is software engineering. A concern with this direction is that sedentary behavior has been found to have a negative effect on health, such as overweight \cite{lakdawalla_labor_2007}, depression \cite{zhai_sedentary_2015}, and a higher risk of cardiovascular events \cite{straker_sedentary_2016}. It is therefore of interested to reduce sedentary behavior while keeping or increasing the productivity of these professions.

There exist previous studies exploring ways to prevent sedentary behavior during work, such as making the employers more aware of their activity with help of notification from a mobile application \cite{cole_they_2015}, introducing standing desks \cite{pronk_reducing_2012}, and walking while having meetings \cite{bort-roig_uptake_2014}. One of these studies found that to make the change be of effects, the intervention needs to not hurt the productivity and be customized for the work context \cite{bort-roig_uptake_2014}. Similarly, this degree project aims to explore the potential to reduce the sedentary behavior for software engineers. But instead of adding intervention to the work environment, this study explores the potential of conducting software engineering in virtual reality that requires movement, reducing the sedentary behavior form the work itself.

In terms of reducing sedentary behavior, virtual reality (VR) that uses six degrees of freedom have its benefits that it typical requires move movement compared sittning in front of a computer. A typical VR-application might need the user to walk, look around, and move their arms to grab or push interfacetbales. Of course, the design of a VR-application heavily effects the amount of these movements. Compare this to a mouse and keyboard setup, where the user only needs to do small movements with their fingers and wrist in order to interact with a GUI. TODO: Give references of positive health effects.

One potential issue with conducting software engineering inside VR is the performance. Software engineering as an occupation often require programming, which is typically done with help of a keyboard. A number of studies have found that typing with VR controllers is slower compared to using a qwerty keyboard [TODO: Cite a number of studies]. Therefore, to generate code by typing inside VR might prevent sedentary behavior, but will also reduce the productivity.

% Change this
One approach that do not require a keyboard by generate code with a visual programming language. That is, you need to move elements around in the program in order to create a desired outcome. There are mainly two ways to implement VPL. One is a block-based approach which is like putting lego together, which is forcing a specific layout of the code blocks [7]. The other is a flow-based approach which is more like putting cables together, and is instead more free regarding placement of the code [7]. VPLs have previously been implemented in VR. For example, FlowMatic [6] is a flow-based environment for creating VR applications, while Cubely [8] makes use of blocks for teaching programming. HackVR [9] combines the flow-based approach with the object oriented paradigm, which they argue has a natural translation to VR applications. However, none of these studies have focused on the health aspect of VPL in VR.



Talk about the potential and goal of this research

The background knowledge required to carry out the project.

\section{RESEARCH QUESTION AND METHOD}

\subsection{Question}
State the question that will be examined. Formulate is as an explicit and evaluable question. State your hypothesis.

\subsection{Objectives}
Objectives: Break down the research questions to measurable objectives.

\subsection{Tasks}
Describe the tasks that are necessary to reach the objectives. For each task, describe the challenges it involves.

\subsection{Method}
Describe the method/s that will be followed. Explain why they are appropriate for the project or for the specific tasks.

\subsection{Ethics and Sustainability}
Does the project address questions of ethics or sustainability? Does the project raise ethical or sustainability questions? If yes, how could these be handled?

\subsection{Limitations}
Define the limitations on what is to be done (so that it is clear what is not included in the degree project).

\subsection{Risks}
Explain what can go wrong and delay or make the project impossible to conclude. Explain how you will deal with these problems.

\section{EVALUATION AND NEWS VALUE}
Evaluation: How is it determined if the objectives of the degree project have been fulfilled and if the research question has been adequately answered? What kind of qualitative or quantitative measures can be defined and evaluated?

Expected scientific results: How is the work scientifically relevant?

The work's innovation/news value. Why does someone want to read the finished work? And who are these people?

\section{PRE-STUDY}
Description of the literature studies. What areas will the literature study focus on? How shall the necessary knowledge on background and state-of-the-art be obtained? What preliminarily important references have been identified?

\section{CONDITIONS AND SCHEDULE}
List of the resources are needed to solve the problem. This can be technical equipment, software, or data, but also experiment and interview subjects.

Describe the way the external supervisor will be involved in the project.

Provide a project timeline, specifying the main tasks and the time allocated for them, milestones (time of achievement of intermediate goals)

% REFERENCES FORMAT
% References must be the same font size as other body text.
\bibliographystyle{SIGCHI-Reference-Format}
\bibliography{References}

\end{document}

%%% Local Variables:
%%% mode: latex
%%% TeX-master: t
%%% End:
